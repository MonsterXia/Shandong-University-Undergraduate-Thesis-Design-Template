% !Mode:: "TeX:UTF-8"

\chapter{山东大学本科毕业论文(设计)撰写规范}

\begin{figure}[htbp]
\centering
\includegraphics[width=\textwidth]{appd.pdf}
\end{figure}
为规范我校本科毕业论文(设计)管理,提升毕业论文(设计)质量,特制定本规范。

本规范约定的书写格式主要适用于用中文撰写的毕业论文(设计)。涉外专业用英文或其他外国语撰写毕业论文(设计)的书写规范可参照本规范执行。各学院和培养单位可根据不同学科、专业特点制定符合本类别毕业论文(设计)的撰写规范。
\section{毕业论文(设计)结构与装订}
毕业论文(设计)一般由以下几部分组成,依次为:封面-成绩评定表-中文摘要(含关键词)-英文摘要(含关键词)-目录-正文-参考文献-致谢-附录-译文中文(可选)-译文原文(可选)-封底。
\section{毕业论文(设计)主要内容及要求}
\subsection{题目}
题目应简洁、精炼、准确且具概括性,能够准确概括论文(设计)的核心内容,一般不超过 20 字,必要时可增加副标题。
\subsection{摘要}
中文摘要应具有高度的概括性,语言精炼、明确,扼要叙述论文(设计)的主要内容,包括研究目的与意义、研究内容与方法以及研究结论等,同时需要突出论文(设计)的新论点、新见解或创造性成果。英文摘要内容应与中文摘要一致,语句通顺,语法正确,准确反映论文(设计)内容。

中文摘要一般约 300-800 个汉字,英文摘要约 200-600 个单词。
\subsection{关键词}
关键词是供检索用的主题词条,应采用能覆盖毕业论文(设计)主要内容的通用技术词条(参照相应的技术术语标准),可从标题或正文中选择 3-5 个最能表达主要内容的词语作为关键词,按词条的外延层次排列(外延大的排在前面)。关键词有中、英文对照,分别附于中、英文摘要后。
\subsection{目录}
目录一般按三级标题编写(如 1、1.1、1.1.1……),层次清晰,与正文中标题一致。
\subsection{正文}
正文包括前言、本论、结论三个部分。
\begin{compactenum}
\item 前言

前言应说明毕业论文(设计)的目的、意义、研究范围及要求达到的技术参数;国内外的发展概况及存在问题;指导思想和应解决的主要问题。
\item 本论

本论是毕业论文(设计)的主体,必须言之成理,论据可靠,严格遵循本学科国际通行的学术规范。写作上要注意结构合理、层次分明、重点突出,章节标题、公式图表符号必须规范统一。

根据不同学科毕业论文(设计)主体的内容及特点,本论一般包括毕业论文(设计)总体方案或选题的论证;各部分的设计实现,如实验数据的获取、数据可行性及有效性的处理与分析、各部分的设计计算等;对研究内容及成果的客观阐述,如理论依据、创新见解、创造性成果及其改进与实际应用价值等。
\item 结论

结论应集中反映作者的研究成果,要求语言精炼、表述准确且完整,着重阐述自己的创造性成果及其在本研究领域中的意义、作用,同时应包括所得结果与已有结果的比较和本课题尚存在的问题,以及进一步开展研究的见解与建议。
\end{compactenum}
\subsection{参考文献}
参考文献是毕业论文(设计)所参考的专著、论文及其他资料(20 篇及以上),所列参考文献应按参考或引证的先后顺序排列,一般应为最新正式发表的文献,其中理、工、医类外文参考文献占比一般不少于 50\%。

注明引用文献的方法通常有三种,即文中注:正文中在引用的地方用括号说明文献出处;脚注:正文中只在引用地方写一个脚注标号,在当页最下方以脚注方式按标号顺序说明文献出处;文末注:在正文引用的地方标号(一般以出现的先后顺序编号,编号以方括号括起,放右上角),然后在全文末 “参考文献”一节,按标号顺序说明文献出处。不同学科可能要求不同,但都应遵循国际上通用习惯以及我国有关国家标准规定,且全文统一,不能混用。
\subsection{致谢}
致谢是对在毕业论文(设计)工作中给予各类资助、指导、协助以及提供各种有利条件的单位、指导教师或其他人员表示感谢,语言应实事求是,切忌浮夸之词。
\subsection{附录}
附录主要包括一些不宜放入正文中的支撑材料,如公式的推演过程、编写的算法、语言程序、各种篇幅较大的图纸等。
\subsection{译文}
中期检查前,学生应完成一篇与毕业论文(设计)课题紧密相关的外文译文(不少于 2000 个汉字)。外文资料由指导教为学生指定,应是一篇完整的选自近期外文书刊的文献,若原文较长,可为文献的核心章节。
\section{毕业论文(设计)的书写和打印规范}
\subsection{文字和字数}
除部分特殊专业外,毕业论文(设计)一律采用国家语言文字工作委员会正式公布的简化汉字书写,英文授课本科专业国际学生的毕业论文(设计)可以采用英文书写,但须附中文摘要。外语类专业毕业论文(设计)采用外语书写。
\subsection{页面设置}
论文(设计)应使用 A4 纸单面或双面纵向打印,上、下页边距 2.5cm,左、右各页边距 3.0cm。

页眉:从正文开始设置,以小五号宋体键入“山东大学本科毕业论文(设计)”,居中显示。

页码:目录页码使用罗马数字(Ⅰ、Ⅱ、Ⅲ)编排;正文页码从正文开始至附录(正文-参考文献-致谢-附录)使用阿拉伯数字编排,小五号 Times New Roman 居中。封面、封底、成绩评定表、中文摘要(含关键词)、外文摘要(含关键词)、外文资料及译文不编入页码。

段前、段后:章标题段前 0.8 行,段后 0.5 行;节标题0.5 行,段后 0.5 行。
\subsection{字体与字号}
\begin{compactenum}
\item 封面

毕业论文(设计)采用山东大学本科毕业论文(设计)统一封面。中文题目用黑体小二号加粗,外文题目用三号黑体字加粗,姓名、学号、学院、年级、指导教师为宋体四号。(英文均采用“Times New Roman”字体)。
\item 中文摘要

“摘要”二字为黑体小二号加粗居中,中间空 4 个空格;容为宋体小四号、1.5 倍行距、首行缩进两字符。
\item 中文关键词

“关键字”三字为黑体小四号加粗;内容为宋体小四号,各关键词用分号(;)隔开,无缩进。
\item 英文摘要

“ABSTRACT”为小二号加粗居中;内容为小四号、1.5 倍行距、首行缩进两字符。
\item 英文关键词

“Key Words”为小四号加粗,内容为小四号,各关键词之间用逗号(,)分开,无缩进。
\item 目录

“目录”二字为黑体小二号加粗居中,空 4 个空格;内容为宋体小四号。
\item 正文

中文一级标题为黑体三号加粗,二级标题为黑体四号加粗,三级及以下标题为黑体小四号加粗;英文一级标题为 15pt 加粗,二级标题为 14pt 加粗;三级及以下标题为 13pt 加粗。

正文内容为宋体小四号、1.5 倍行距、首行缩进 2 字符。
\item 参考文献

“参考文献”四字为黑体小二号加粗居中;内容为宋体五号、单倍行距、首行无缩进。
\item 致谢

“致谢”二字为黑体小二号加粗居中,中间空 4 个空格;内容为宋体小四号、1.5 倍行距、首行缩进两字符。
\item 附录

“附录”二字为黑体小二号加粗居中,中间空 4 个空格;内容为宋体小四号、首行缩进两字符、1.5 倍行距。
\item 译文

译文标题为黑体小二号加粗居中,内容为宋体小四号、1.5倍行距、首行缩进两字符。外文原文标题为小二号加粗居中,内容小四号、1.5 倍行距、首行缩进两字符,或直接使用原文 PDF版本。
\end{compactenum}
\subsection{标题层次}
毕业论文(设计)的全部标题层次应有条不紊,整齐清晰,相同的层次应采用统一的表示体例,正文中各级标题下的内容应同各自的标题对应,不应有与标题无关的内容。

理学、工学、医学类学位论文(设计)的章节编排建议遵循《CY/T 35-2001 科技文献的章节编号方法》,以阿拉伯数字编号,如 1,1.2,1.2.1 逐级递推。人文社科类学位论文(设计)建议采用 第一章 第一节 一、 (一) 形式编排。英文撰写的文(设计)建议采用 Chapter1 ,1.2, 1.2.1 逐级递推。
\subsection{图片、表格及公式}
\begin{compactenum}
\item 图片

毕业论文(设计)的插图应与文字紧密配合,文图相符,技术内容正确。选图要力求精练,线条要匀称,图面要整洁美观,居中,每幅插图应有图序和图题(宋体五号加粗居中),全文插图可以统一编序,也可以每章单独编序(如图 1.1 XXX,图 2.1XXX),不管采用哪种方式,图序必须连续,不得重复或跳缺。由若干分图组成的插图,分图用 a、b、c……标序,分图的图名以及图中各种代号的意义,以图注形式写在图题下方,先写分图名,另起一行后写代号的意义。

图应在描纸或洁白纸上用墨线绘成,或用计算机绘图,电气图或机械图应符合相应的国家标准。坐标图:横纵坐标必须标注物理量、单位,坐标名置于图的下方居中,宋体五号加黑。

图应放在离正文首次出现处的近处,不应过分超前或拖后。
\item 表格

每个表格应有自己的表序和表题,位于表格上方正中,表序后不加标点,空一格后写表题,表题末尾不加标点。表题用宋体五号加粗居中,表格内中文用宋体五号,英文用 Times New Roman,字体五号,表格格式采用简明三线表。

全文的表格可以统一编序,也可以每章单独编序(如表 1-XXX,表 2-1 XXX),不管采用哪种方式,表序必须连续。表格允许下页接写,接写时表题省略,表头应重复书写,并在右上方写“续表××”。此外,表格应放在离正文首次出现处的近处,不应过分超前或拖后。

\item 公式

公式应另起一行居中对齐,一行写不完的长公式,最好在等号处转行,如做不到这点,应在数学符号(如“+”、“-”号)处转行,且数学符号应写在转行后的行首。

公式的编号用圆括号括起放在公式右边行末,靠右对齐,公式和编号之间不加虚线,公式可按全文统一编序号,也可以每章单独编序(如第 1 章中第一个公式编号为 1-1,第 2 章中第二公式编号为 2-2),公式序号必须连续,不得重复或跳缺。重复引用的公式不得另编新序号。公式中分数的横分线要写清楚,特别是连分数(即分子和分母也出现分数时)更要注意分线的长短,并将主要分线和等号对齐。在叙述中也可将分数的分子和分母平列在一行,用斜线分开表述。
\end{compactenum}
\subsection{量和单位}
毕业论文(设计)中的量和单位必须采用中华人民共和国家标准 GB 3100~GB 3102-1993,它是以国际单位制(SI)为基础的。非物理量的单位,如件、台、人、元等,可用汉字与符号构成组合形式的单位,例如:件/台、元/km。
\subsection{标点符号、数字}
标点符号应按《标点符号用法》(中华人民共和国国家标准GB/T15834-2011)使用。测量、统计数据一律用阿拉伯数字,如5.25MeV 等。在叙述不是特别大的数目时,一般不宜用阿拉伯数字。
\subsection{名词、名称}
科学技术名词术语尽量采用全国科学技术名词审定委员会公布的规范词或国家标准、部标准中规定的名称。尚未统一规定或叫法有争议的名词术语,可采用惯用的名称。使用外文缩写代替某一名词术语时,首次出现时应在括号内注明其含义,如:OECD(Organization for Economic Cooperation and Development)代替经济合作发展组织。

外国人名一般采用外文原名,可不译成中文,英文人名按姓前名后的原则书写,如:CRAY P.,不可将外国人姓名中的名部分漏写,例如:不能只写 CRAY, 应写成 CRAY的外国人名(如牛顿、爱因斯坦、达尔文、马克思等)可按通常标准译法写译名。
\subsection{参考文献著录格式示例}
书写格式应符合 GB/T 7714-2015《信息与文献 参考文献著录规则》。

\vspace{2cm}

\begin{figure}[htbp]
\centering
\includegraphics[width=\textwidth]{print.pdf}
\end{figure}